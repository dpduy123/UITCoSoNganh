\clearpage
\section{Redirecting Input and Output}
\subsection{Yêu cầu cụ thể}
\subsubsection{Đề bài}
\begin{tcolorbox}[
  colback=blue!5!white, 
  colframe=blue!75!black, 
  title=Task 3, 
  attach boxed title to top center={
    yshift=-2mm
  },
  boxrule=0.5mm,
  width=\textwidth,
  bottomtitle=0.5mm
]
Your shell should then be modified to support the ‘>’ and ‘<’ redirection
operators, where ‘>’ redirects the output of a command to a file and ‘<’ redirects 
the input to a command from a file. For example, if a user enters
\begin{center}
    \texttt{osh>ls > out.txt}
\end{center}

the output from the \texttt{ls} command will be redirected to the file \texttt{out.txt}. Similarly, input can be redirected as well. For example, if the user enters
\begin{center}
\texttt{osh>sort < in.txt}
\end{center}
the file \texttt{in.txt} will serve as input to the \texttt{sort} command.
Managing the redirection of both input and output will involve using the
\texttt{dup2()} function, which duplicates an existing file descriptor to another file
descriptor. For example, if \texttt{fd} is a file descriptor to the file \texttt{out.txt}, the call
\begin{center}
\texttt{dup2(fd, STDOUT\_FILENO);}
\end{center}
duplicates \texttt{fd} to standard output (the terminal). This means that any writes to
standard output will in fact be sent to the \texttt{out.txt} file.
You can assume that commands will contain either one input or one output
redirection and will not contain both. In other words, you do not have to be
concerned with command sequences such as \texttt{sort < in.txt > out.txt}.

\begin{flushright}
\textit{Page 201, Operating System Concepts 10th Edition}
\end{flushright}

\end{tcolorbox}

\subsubsection{Mô tả đề bài}
Đề yêu cầu ta cung cấp cho giao diện shell chức năng hỗ trợ các toán tử chuyển hướng \texttt{$>$} và \texttt{$<$}, trong đó \texttt{$>$} chuyển hướng đầu ra của lệnh vào một tệp, và \texttt{$<$} chuyển hướng đầu vào của lệnh từ một tệp. Ví dụ, lệnh \texttt{ls $>$ out.txt} sẽ chuyển đầu ra của lệnh \texttt{ls} vào tệp \texttt{out.txt}, còn lệnh \texttt{sort $<$ in.txt} sẽ sử dụng nội dung của tệp \texttt{in.txt} làm đầu vào cho lệnh \texttt{sort}. Việc quản lý chuyển hướng đầu vào và đầu ra được thực hiện bằng cách sử dụng hàm \texttt{dup2()} để ánh xạ một file descriptor hiện tại tới một file descriptor khác, chẳng hạn như ánh xạ đầu ra chuẩn (\texttt{STDOUT\_FILENO}) tới một tệp. Đề bài giả định rằng mỗi lệnh chỉ chứa một toán tử chuyển hướng đầu vào hoặc đầu ra, và không cần xử lý các lệnh chứa cả hai loại chuyển hướng.

\subsection{Hướng giải quyết}
Để giải quyết yêu cầu đề ra, ta chia thành 2 vấn đề nhỏ. Đầu tiên, ta phải xác định xem các tệp ánh xạ đầu ra và đầu vào liệu có trục trặc gì không, nếu có phải đưa ra thông báo và dừng chương trình, đồng thời ánh xạ các tệp đầu vào/ra vào đầu vào/ra chuẩn tương ứng $\left(\text{STDIN\_FILENO}/\text{STDOUT\_FILENO}\right)$ . Sau đó ta mới bắt đầu đi vào thực hiện câu lệnh chứa toán tử chuyển hướng. Ta đi viết hai hàm: \texttt{handleRedirection(char **args)} và \\ \texttt{executeCommandWithRedirection(char **args, int background)} để giải quyết lần lượt các vấn đề . Trong đó, \texttt{**args} là câu lệnh đầu vào. Hai hàm được mô tả ở mục \textbf{3} và \textbf{4} cùng chương.

\subsection{Hàm handleRedirection}
\subsubsection{Input}
Hàm nhận vào một mảng các chuỗi ký tự \texttt{args}, trong đó mỗi phần tử là một phần của lệnh người dùng nhập vào. Mảng này có thể bao gồm các toán tử chuyển hướng đầu vào (\texttt{<}) và đầu ra (\texttt{>}) cùng với tên tệp cần chuyển hướng.

\subsubsection{Output}
Hàm trả về một giá trị nguyên \texttt{-1} nếu có lỗi trong quá trình chuyển hướng (ví dụ: thiếu tệp đầu vào hoặc đầu ra)  và hiển thị thông báo lỗi. Nếu không, hàm không trả về giá trị nguyên nhưng sẽ thay đổi các file descriptor chuẩn (stdin, stdout) để chuyển hướng đầu vào hoặc đầu ra của lệnh trong quá trình hoạt động. 

\subsubsection{Hướng giải thuật giải quyết}
Hàm \texttt{handleRedirection} thực hiện các bước sau để xử lý chuyển hướng đầu vào và đầu ra:
\begin{enumerate}
    \item \textbf{Lưu lại các file descriptor gốc}: Sử dụng \texttt{dup(STDIN\_FILENO)} và \texttt{dup(STDOUT\_FILENO)} để lưu lại các file descriptor chuẩn đầu vào và đầu ra.
    \item \textbf{Duyệt qua các tham số lệnh}: Duyệt qua mảng \texttt{args} để tìm các toán tử chuyển hướng \texttt{<} và \texttt{>}.
    \item \textbf{Xử lý toán tử chuyển hướng \texttt{>}}: Nếu gặp toán tử \texttt{>}, hàm sẽ kiểm tra xem tệp đích có tồn tại hay không. Nếu không có tệp, hàm sẽ tạo tệp mới. Sử dụng \texttt{open()} để mở tệp và \texttt{dup2()} để chuyển hướng đầu ra vào tệp.
    \item \textbf{Xử lý toán tử chuyển hướng \texttt{<}}: Nếu gặp toán tử \texttt{<}, hàm sẽ kiểm tra xem tệp nguồn có tồn tại không. Nếu không có tệp, hàm sẽ báo lỗi. Sử dụng \texttt{open()} để mở tệp và \texttt{dup2()} để chuyển hướng đầu vào từ tệp.
    \item \textbf{Kết quả}: Nếu không có lỗi, lệnh sẽ được thực thi với đầu vào/đầu ra đã được chuyển hướng.
    
\end{enumerate}

\subsubsection{Lưu đồ thuật toán}
\begin{figure}
    \centering
    \includegraphics[width=1\linewidth]{handredirec.png}
\end{figure}
\newpage

\subsubsection{Code}
\begin{lstlisting}
int handleRedirection(char **args) {
    int fd;
    int stdin_backup = dup(STDIN_FILENO);  
    int stdout_backup = dup(STDOUT_FILENO); 

    for (int i = 0; args[i] != NULL; i++) {
        if (strcmp(args[i], ">") == 0) {
            if (args[i + 1] == NULL) {
                fprintf(stderr, "Error: Missing output file.\n");
                return -1;
            }
            fd = open(args[i + 1], O_CREAT | O_WRONLY | O_TRUNC, S_IRWXU);
            if (fd == -1) {
                perror("Error opening output file");
                return -1;
            }
            dup2(fd, STDOUT_FILENO);
            close(fd);
            args[i] = NULL;
        } else if (strcmp(args[i], "<") == 0) {
            if (args[i + 1] == NULL) {
                fprintf(stderr, "Error: Missing input file.\n");
                return -1;
            }
            fd = open(args[i + 1], O_RDONLY);
            if (fd == -1) {
                perror("Error opening input file");
                return -1;
            }
            dup2(fd, STDIN_FILENO);
            close(fd);
            args[i] = NULL;
        }
    }

    return 0;
}
\end{lstlisting}

\subsection{Hàm executeCommandWithRedirection}
\subsubsection{Input}
Hàm nhận vào mảng các chuỗi ký tự \texttt{args}, đại diện cho lệnh và các tham số, và một biến \texttt{background} cho biết lệnh có được thực thi trong nền hay không.

\subsubsection{Output}
Hàm không trả về giá trị nhưng thực hiện việc chuyển hướng đầu vào/đầu ra (nếu có) và thực thi lệnh. Sau khi lệnh được thực thi, các file descriptor chuẩn (stdin, stdout) sẽ được khôi phục lại trạng thái ban đầu.

\subsubsection{Hướng giải thuật giải quyết}
Hàm \texttt{executeCommandWithRedirection} thực hiện các bước sau:
\begin{enumerate}
    \item \textbf{Lưu lại các file descriptor gốc}: Sử dụng \texttt{dup(STDIN\_FILENO)} và \texttt{dup(STDOUT\_FILENO)} để lưu lại các file descriptor gốc của đầu vào và đầu ra.
    \item \textbf{Xử lý chuyển hướng}: Gọi hàm \texttt{handleRedirection(args)} để xử lý các toán tử chuyển hướng (\texttt{<} và \texttt{>}). Nếu có lỗi trong quá trình chuyển hướng (ví dụ: thiếu tệp), hàm sẽ khôi phục lại các file descriptor gốc và dừng việc thực thi.
    \item \textbf{Thực thi lệnh}: Nếu chuyển hướng thành công, gọi hàm \texttt{executeCommand(args, background)} để thực thi lệnh với đầu vào và đầu ra đã được chuyển hướng (nếu có).
    \item \textbf{Khôi phục lại các file descriptor}: Sau khi lệnh được thực thi, sử dụng \texttt{dup2()} để khôi phục lại các file descriptor gốc cho đầu vào và đầu ra.
    \item \textbf{Kết quả}: Sau khi khôi phục lại các file descriptor, hàm sẽ kết thúc và lệnh được thực thi đúng cách với chuyển hướng đầu vào/đầu ra (nếu có).
\end{enumerate}
\subsubsection{Lưu đồ thuật toán}
\begin{figure}[H]
    \centering
    \includegraphics[width=0.75\linewidth]{redirection.png}
\end{figure}
\newpage
\subsubsection{Code}
\begin{lstlisting}
void executeCommandWithRedirection(char **args, int background) {
    int stdin_backup = dup(STDIN_FILENO);  
    int stdout_backup = dup(STDOUT_FILENO); 

    if (handleRedirection(args) == -1) {
        dup2(stdin_backup, STDIN_FILENO);  
        dup2(stdout_backup, STDOUT_FILENO); 
        close(stdin_backup);
        close(stdout_backup);
        return;
    }

    executeCommand(args, background);
    dup2(stdin_backup, STDIN_FILENO);
    dup2(stdout_backup, STDOUT_FILENO);
    close(stdin_backup);
    close(stdout_backup);
}
\end{lstlisting}
